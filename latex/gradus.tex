% mnras_template.tex 
%
% LaTeX template for creating an MNRAS paper
%
% v3.0 released 14 May 2015
% (version numbers match those of mnras.cls)
%
% Copyright (C) Royal Astronomical Society 2015
% Authors:
% Keith T. Smith (Royal Astronomical Society)

% Change log
%
% v3.0 May 2015
%    Renamed to match the new package name
%    Version number matches mnras.cls
%    A few minor tweaks to wording
% v1.0 September 2013
%    Beta testing only - never publicly released
%    First version: a simple (ish) template for creating an MNRAS paper

%%%%%%%%%%%%%%%%%%%%%%%%%%%%%%%%%%%%%%%%%%%%%%%%%%
% Basic setup. Most papers should leave these options alone.
\documentclass[fleqn,usenatbib]{mnras}

% MNRAS is set in Times font. If you don't have this installed (most LaTeX
% installations will be fine) or prefer the old Computer Modern fonts, comment
% out the following line
\usepackage{newtxtext,newtxmath}
% Depending on your LaTeX fonts installation, you might get better results with one of these:
%\usepackage{mathptmx}
%\usepackage{txfonts}

% Use vector fonts, so it zooms properly in on-screen viewing software
% Don't change these lines unless you know what you are doing
\usepackage[T1]{fontenc}

% Allow "Thomas van Noord" and "Simon de Laguarde" and alike to be sorted by "N" and "L" etc. in the bibliography.
% Write the name in the bibliography as "\VAN{Noord}{Van}{van} Noord, Thomas"
\DeclareRobustCommand{\VAN}[3]{#2}
\let\VANthebibliography\thebibliography
\def\thebibliography{\DeclareRobustCommand{\VAN}[3]{##3}\VANthebibliography}


%%%%% AUTHORS - PLACE YOUR OWN PACKAGES HERE %%%%%

% Only include extra packages if you really need them. Common packages are:
\usepackage{graphicx}	% Including figure files
% \usepackage{amsmath}	% Advanced maths commands
% \usepackage{amssymb}	% Extra maths symbols (this seems to be already defined)
\usepackage{cuted}
\setlength{\stripsep}{0ex}

\usepackage{tcolorbox}
\usepackage{listings}
%%
%% Julia definition (c) 2020 Daichi Furukawa
%%
\lstdefinelanguage{Julia}{%
  morekeywords={abstract, ans, baremodule, begin, break, const, continue, do,%
    elseif, else, end, export, finally, for, function, global, if, import,%
    let, local, macro, module, mutable, primitive, quote, return, struct, try,%
    type, using, var, where, while},%
  % Types
  morekeywords=[2]{Any, Array, Bool, BigInt, BigFloat, Char, DataType, Enum,%
    Expr, Float16, Float32, Float64, Function, Inf, Inf16, Inf32, Int8, Int16,%
    Int32, Int64, Int128, Matrix, Missing, Module, Nan, NaN16, NaN32, Nothing,%
    Some, Symbol, Tuple, Type, UInt8, UInt16, UInt32, UInt64, UInt128, Union,%
    UnionAll, Val, Vararg, Vector, VecOrMat, im},%
  % Constants
  morekeywords=[3]{ARGS, C_NULL, DEPOT_PATH, ENDIAN_BOM, ENV, LOAD_PATH,%
    PROGRAM_FILE, Sys.ARCH, Sys.BINDIR, Sys.CPU_THREADS, Sys.KERNEL,%
    Sys.MACHINE, Sys.WORD_SIZE, false, stderr, stdin, stdout, true, missing, nothing},%
  % Macros
  morekeywords=[4]{@NamedTuple, @allocated, @assert, @boundscheck, @deprecate,%
    @elapsed, @enum, @eval, @generated, @gensym, @goto, @isdefined, @inbouds,%
    @inline, @label, @noinline, @nospecialize, @polly, @propagate_inbounds,%
    @pure, @specialize, @simd, @time, @timed, @timev, @static, @v_str, @view,%
    @views},%
  sensitive=true,%
  morecomment=[l]\#,%
  morestring=[b]',%
  morestring=[b]",%
  morecomment=[s]{"""}{"""},% used for documentation text
                            % (mulitiline strings)
  morestring=[s]{r"}{"},% regular expression literal
  morestring=[s]{r"""}{"""},%
  morestring=[s]{raw"}{"},% raw string literals
  morestring=[s]{raw"""}{"""},%
  morestring=[s]{L"}{"},% LaTeX strings
  morestring=[s]{L"""}{"""}%
}%

\lstset{
    frame=single,
    % frameround=tttt,
    xleftmargin=2.4em,
    xrightmargin=1.3em,
    aboveskip=3mm,
    belowskip=1.0mm,
    backgroundcolor=\color{gray!3},
    showstringspaces=false,
    columns=flexible,
    basicstyle={\small\ttfamily},
    numbers=left,
    breaklines=true,
    breakatwhitespace=true,
    tabsize=4,
    keywordstyle = \bfseries\color{blue},
    stringstyle = \color{magenta},
    commentstyle = \color{green!70!black},
    language=Julia,
}

%%%%% AUTHORS - PLACE YOUR OWN COMMANDS HERE %%%%%

\newcommand{\todo}[1]{{\bf \color{red} #1}}
\newcommand{\notes}[1]{{\color{cyan} #1}}

\newcommand{\Gradus}{Gradus.jl}

\newcommand{\e}{\text{e}}
\renewcommand{\d}{\text{d}}
\newcommand{\utensor}[3]{#1^{#2}_{\phantom{#2}#3}}
\newcommand{\dtensor}[3]{#1_{#2}^{\phantom{#2}#3}}
\newcommand{\stensor}[3]{#1_{#2}^{#3}}
\newcommand{\deriv}[2]{\frac{\d #1}{\d #2}}
\newcommand{\pderiv}[2]{\frac{\partial #1}{\partial #2}}

%%%%%%%%%%%%%%%%%%% TITLE PAGE %%%%%%%%%%%%%%%%%%%

\title[Gradus.jl]{Gradus.jl: extensible and spacetime agnostic general relativistic ray-tracing for reverberation modelling through automatic differentiation}

% The list of authors, and the short list which is used in the headers.
% If you need two or more lines of authors, add an extra line using \newauthor
\author[F. J. E. Baker et al.]{
F. J. E. Baker,$^{1}$\thanks{E-mail: fergus.baker@bristol.ac.uk (FB)}
and A. J. Young$^{1}$
% Plus other contributing authors (tbc)
\\
% List of institutions
$^{1}$H. H. Wills Physics Laboratory, Tyndall Avenue, Bristol BS8 1TL, UK
}

% These dates will be filled out by the publisher
\date{Accepted XXX. Received YYY; in original form ZZZ}

% Enter the current year, for the copyright statements etc.
\pubyear{2023}

% Don't change these lines
\begin{document}
\label{firstpage}
\pagerange{\pageref{firstpage}--\pageref{lastpage}}
\maketitle

% Abstract of the paper
\begin{abstract}
	We introduce \Gradus, an open-source...
\end{abstract}

% Select between one and six entries from the list of approved keywords.
% Don't make up new ones.
\begin{keywords}
keyword1 -- keyword2 -- keyword3
\end{keywords}

%%%%%%%%%%%%%%%%%%%%%%%%%%%%%%%%%%%%%%%%%%%%%%%%%%

%%%%%%%%%%%%%%%%% BODY OF PAPER %%%%%%%%%%%%%%%%%%

%% INTRODUCTION
\section{Introduction}

\notes{
In the era of quantitative, precision observational tests of General Relativity in the strong field regime it is necessary to have a fast and flexible method to compute the observational properties of accreting black hole systems. We have developed an open-source integrator 
\Gradus\footnote{Open-source and available under MIT license at \url{https://github.com/astro-group-bristol/Gradus.jl}.} for this purpose. In the remainder of the paper we describe how the software works, comparing with previous work in the literature, and outlining the new capabilities of \Gradus.

% Maybe some history of the problem with key references
Transfer functions \citep{cunningham_effects_1975} % An example reference

Julia is a high-performance... with SciML and DifferentialEquations.jl, a state-of-the-art ecosystem and workhorse for solving differential equations. 
}

\todo{goerge and fabian for reflection of xrays off cold ad}

\subsection{A short review of general relativistic ray tracing}

\todo{
\begin{itemize}
    \item review other softwares
    \item where does Gradus fit into the field (extensibility / maintainability / prototyping models)
    \item define SSD as Shakura Sunyaev disc
\end{itemize}
}

The trajectory of light in curved space may be determined by reformulating the Hamilton-Jacobi equations of motion as a first-order ordinary differential equation (ODE) system. 

A second-order ODE system may alternatively be formulated directly from the geodesic equation; a method which is pedagogically simpler, but computationally more expensive than the first-order system, as either the full metric connection or derivatives of the metric must be explicitly implemented, else approximated at cost during runtime. With advancements in automatic differentiation, derivatives are cheap to compute, and consequently the second-order approach is tractable and both a parsimonious and spacetime agnostic method for computing geodesics.


%% NUMERICAL METHODS
\section{Numerical methods}

For simplicity, we will here focus on static, axisymmetric spacetimes in the Boyer-Lindquist coordinates. Such a spacetime has a metric of the form 
\begin{equation}
\label{eq:static_axisymmetric_metric}
    g_{\mu\nu} 
    = g_{tt} \d t^2 
    + g_{rr} \d r^2 
    + g_{\theta\theta} \d \theta^2 
    + g_{\phi\phi} \d \phi^2 
    + 2g_{t\phi} \d t \d \phi.
\end{equation}
We adopt $(-, +, +, +)$ signature and standard units $c = G = 1$. Greek indices ($\mu, \nu$) will be used to denote the four spacetime components, and Latin indices ($i, j$) for the three spatial components. We write partial derivatives with respect to the coordinates $x^\mu$ as $\partial_\mu := \partial / \partial x^\mu$.


%% SECTION: GEODESIC INTEGRATION
\subsection{Geodesic integration}

The geodesic equation with coordinates $x^\mu$ is
\begin{equation}
\label{eq:geodesic_equation}
    \frac{\d^2 x^\mu}{\d \lambda^2}
    + \utensor{\Gamma}{\mu}{\nu\sigma}
    \vel{\nu}
    \vel{\sigma}
    = a^\mu,
\end{equation}
where $\lambda$ is the affine parameter and $a^\mu$ is some external acceleration. The effects of curvature on the trajectory encoded in the Christoffel symbols, 
\begin{equation}
\label{eq:christoffel}
    \utensor{\Gamma}{\mu}{\nu\sigma}
    := \frac{1}{2} g^{\mu\rho} 
    \left(
        \partial_{\nu}g_{\rho \sigma}
        + \partial_{\sigma}g_{\rho \nu}
        - \partial_{\rho}g_{\sigma \nu}
    \right),
\end{equation}
determined solely by metric and derivatives thereof. The geodesic equation is a set of four coupled second order differential equations that may be solved for a choice of initial $x^\mu$ and $\vel{\mu}$. A convention is to choose an initial position with $x^t = 0$, whereas the velocity vector is additionally constrained by the invariance
\begin{equation}
\label{eq:velocity_constraint}
    g_{\sigma\nu} \vel{\sigma} \vel{\nu} = \mu^2,
\end{equation}
where $\mu$ is the invariant mass. This invariance gives rise to three solution classes depending on the sign of $\mu^2$, namely $\mu^2 = 0$ corresponding to null-, $\mu^2 > 0$ to time-, and $\mu^2 < 0$ to space-like geodesics. Null geodesics are the trajectories of photons, time-like regular massive particles, and space-like geodesics are the trajectories of exotic particles, such as tachyons. Specifying the three-vector $\vel{i}$ determines $\vel{t}$ by Eq. \eqref{eq:velocity_constraint}, rearranged as
\begin{equation}
\vel{t}  = \frac{-g_{t\phi} \vel{\phi} \pm
    \sqrt{-g_{ij} \vel{i} \vel{j} - \mu^2}
}{g_{tt}}.
\end{equation}
The choice of positive or negative root corresponds to the direction of time, wherein lies the ray-tracing \textit{trick}: a time-reversal symmetry in the metric allows us to trace from an observer back to the point of origin, and then \textit{reverse} time in order to calculate quantities as seen by the observer. \todo{explain better}

The integration of the ODE system is performed numerically with an appropriate choice of algorithm. We favour the adaptive Tsitouras Runge-Kutta 5/4 \citep{tsitouras_rungekutta_2011}. This integrator is shown to be fast and robust, also providing free fourth-order interpolants of the resulting geodesics. The interpolants can be particularly useful if additional quantities need to be re-traced along a geodesic, or for accurately finding points of intersections.

Derivatives of the metric needed in Eq. \eqref{eq:christoffel} may be efficiently computed with AD. This brings versatility and ease, as new spacetimes need only to define the metric for the geodesic system to be computable, and AD ensures the derivatives are free of the pathologies of other e.g. stenciling methods used to compute derivatives. If the class of spacetime exhibits additional symmetries, these can be exploited to reduce computation further: metrics of the form \eqref{eq:static_axisymmetric_metric} can exploit $\partial_t g_{\mu\nu} = \partial{\phi} g_{\mu\nu} = 0$ and avoid calculating two columns of the Jacobian entirely.

Sparsity is often symbolically inferable under simple operations. We use compile time Julia \texttt{@generated} functions, along with Symbolics.jl, to attempt to infer which terms in the Jacobian and Christoffel symbols can be avoided (if any), to generate optimal evaluation for a given class of metric. 
%This was benchmarked against other packages that perform the sparsity detection generically, such as FastDifferentiation.jl, and find similar performance.


\subsection{Observers and emitters}

To determine interpretable initial velocities it is useful to consider the coordinates local to the observer or emitter at coordinates $x^\mu$. Following \cite{cunningham_optical_1973}, for observers one usually considers an image plane onto which a projection of geodesics is rendered, with geometry as in Fig. \ref{fig:observer_coordinates}. One may then define a set of impact parameters from components of the local momenta
\begin{align}
    \alpha &:=  x^r \frac{p_{(\phi)}}{p_{(r)}}, \\
    \beta &:= x^r \frac{p_{(\theta)}}{p_{(r)}},
\end{align}
using indices in parentheses to denote the vector components in the local frame. Our choice of symbols ($r, \theta, \phi$) is to anticipate an identification between local and global bases, and to provide some intuition for the local frame.

Along with the invariance of four-momenta (c.f. Eq. \eqref{eq:velocity_constraint}), one may obtain a curve of solutions for the local momenta
\begin{align}
    \frac{p_{(r)}}{p_{(t)}} &= -\left( \sqrt{1 + \left(\frac{\alpha}{x^r}\right)^2 + \left(\frac{\beta}{x^r}\right)^2} \right)^{-1} = \mathscr{R}, \\
    \frac{p_{(\theta)}}{p_{(t)}} &= \mathscr{R} \frac{\beta}{x^r}, \\
    \frac{p_{(\phi)}}{p_{(t)}} &= \mathscr{R} \frac{\alpha}{x^r}.
\end{align}
For emitters, the above set of local initial momenta may be determined from a tangent vector pointing along the direction of a geodesic by decomposition (see also \ref{fig:observer_coordinates}). The component $p_{(t)}$ is the negative of the energy measured in the local frame, and is conventionally set to $-p_{(t)} = E = 1$ without loss of generality. 

For both observers and emitters, we must identify a local frame, where the natural choice is the locally non-rotating frame (LNRF) \citep{bardeen_rotating_1972} \todo{check citation + explain significance}. The transformation from local to global coordinates is
\begin{equation}
    p_\mu = \e^{(\nu)}_{\phantom{(\nu)}\mu}\  p_{(\nu)}
\end{equation}
where the basis vectors $\e^{(\nu)}_{\phantom{(\nu)}\mu}$ are found using the theorem of Gram-Schmidt (\cite{}, Appendix \ref{appendix:gram-schmidt}). The formalism may be extended for an observer or emitter in motion, where their velocity modifies the mapping by a local Lorenz transformation, $\Lambda^{(\kappa)}_{\phantom{(\kappa)}(\nu)}$, as 
\begin{equation}
    p_\mu = \e^{(\nu)}_{\phantom{(\nu)}\mu}\  \Lambda^{(\kappa)}_{\phantom{(a)}(\nu)} p_{(\kappa)}.
\end{equation}

\subsection{Charts and horizons}

A chart is used to terminate geodesic integration to avoid continuing computation when the fate of a given geodesic is determined. In practical terms, the chart is defined by a set of boundaries, and used to classify the outcome of an integration. As a motivating example, consider a chart with an inner and outer boundary: the inner boundary is a coordinate singularity of the metric, $r_s$ (i.e. an event horizon), whereas the outer boundary is treated as the \textit{effective infinity}, $r_\infty$. Geodesics at the inner boundary are classified as lost behind the coordinate singularity, whereas those that reach the outer boundary are considered to escape to infinity with no further deviation to their trajectory. Additional boundaries of the chart may be used to represent accretion geometry: by terminating the integration when the geodesic crosses such a boundary, we consider the geodesic to have intersected the surface of the accretion disc. 

The event horizon radius, $r_s$, used as the default inner boundary, may be calculated for a metric of the form \eqref{eq:static_axisymmetric_metric} by solving
\begin{equation}
    \label{eq:event_horizon}
    0 = \left. \frac{1}{g_{rr}} \right\rvert_{x^r = r_s}.
\end{equation}
For axisymmetric metrics, $g_{rr}$ may be a function of both $x^r$ and $x^\theta$, in which case the inner boundary of the chart is a function of the poloidal coordinate. If no analytic function for $r_s$ is known, it may be numerically approximated using root solving methods. We use root solvers from Roots.jl \citep{}. The default we use is their ``Order 0'' solver, a hybrid method that refines from secant to bracketing methods when possible.

In practice, close to the inner radius the adaptive time step of an ODE integrator tends to shrink dramatically due to near-singular derivatives, causing the integration to slow to almost a standstill. This may be avoided by scaling the inner horizon with the choice of constant $\mathcal{K} > 0$, such that $\tilde{r}_s = (1 + \mathcal{K}) r_s$, terminating the integration early when $x^r \leq \tilde{r}_s$. This constant may be adjusted depending on how vital it is for a geodesic to be able to glance the event horizon. We have chosen $\mathcal{K} = 10^{-2}$ by default, as it dramatically improves integration time without impacting the majority of simulations.


\subsection{Computing observables}

We consider observables to be any physical quantity calculated from a simulation that is evaluated using some or all of the points along a geodesic. Often only the start and end point of a geodesic are required to calculate some physical quantity, and under such circumstances it is computational beneficial to avoid allocating space for the full solutions. 

A useful quantity to compute is the apparent redshift along a geodesic, due to both the Doppler and gravitational redshift. This may be compactly written as the ratio of energies
\begin{equation}
g = \frac{E_\text{end}}{E_\text{start}} = \frac{\left. k_\mu u^\mu \right\rvert_\text{end}}{\left. k_\mu u^\mu \right\rvert_{\text{start}}},
\end{equation}
where $k^\mu$ are the photon momenta, and $u^\mu$ the velocity of the emitting (start) and observing (end) media respectively.

\subsection{Solving for special orbits}

For simple disc models, including the $\alpha$-disc of \cite{shakura_black_1973}, the accreting matter is considered to follow Keplerian circular orbits in the equatorial plane, $x^\theta = \pi/2$, of the central singularity. These orbits are stationary points of the Hamiltonian and constrained by $v^r = v^\theta = 0$. For the class of static, axis-symmetric methods for particles with no external forces, they may be analytically determined (\cite{johannsen_regular_2013}, with a variation of their derivation in Appendix).

In other cases, such when the external acceleration $a^\mu \neq 0$ in \eqref{eq:geodesic_equation}, the analytic approach is often useful up to a point, before resolving to numerical methods for root finding.

Orbits may also be determined purely from the integration of geodesics, by mandating a stability measure and optimizing the initial velocity vector until the measure is a minimum. For example, let $\mathscr{M}$ measure the eccentricity of an orbit in the equatorial plane. For a given radius $x^r$, the velocity corresponding to a circular orbit is $v^\mu = v^t \partial_t + v^\phi \partial_\phi $. With \eqref{eq:velocity_constraint}, the velocity may be found through
\begin{equation}
    \underset{v^\phi}{\arg \min}\ \ \mathscr{M}(x^r, v^\phi),
\end{equation}
using a numerical optimizer. We use Nelder-Mead as the default algorithm. This optimization method is also used for arbitrary objective functions, such as for pinpointing geodesics that intersect chosen points ($\mathscr{M}$ measures closest approach) or exhibit specific desired features (e.g. $\mathscr{M}$ measures periodicity).

Circular orbits are classified as either stable or unstable, depending on the sign of $\d E / \d x^r$, with $>0$ corresponding to stable configurations. Stable circular orbits are only possible for radii above the innermost stable circular orbit (ISCO) radius, $x^r \geq r_\text{ISCO}$. The ISCO is the critical point at which 
\begin{equation}
    0 = \left. \frac{\d E}{\d x^r} \right\rvert_{x^r = r_\text{ISCO}}.
\end{equation}
Within the ISCO is the so-called \textit{plunging region} where $v^r \neq 0$. Circular orbits in this region are highly unstable and will either fall into the event horizon or escape to infinity if perturbed. These orbits may be numerically calculated from the ISCO four-velocity by offsetting $r_\text{ISCO} -  \delta x^r$, and integrating over a large $\lambda$ interval. The components of $v^\mu$ may then be interpolated over $x^r$ to approximate analytic solutions.


\notes{
How we find circular orbits, how we find ISCO, photon radius, event horizon
}

\subsection{Disc emissivity}

\notes{
We can calculate emissivity / flux maps for discs using either Voronoi tesselation or some symmetric prescriptions
}

\subsection{Transfer functions}


\notes{
Both 1d (Cunningham) and 2d (lag-energy) transfer functions, the methods used to solve them, and the quadrature integration schemes.
}

\subsection{Covariant radiative transfer}
From \eqref{eq:geodesic_equation}, with $a^\mu = f^\mu / m$.

WIP? will probably finish implementing this before writing the full paper



%% DESCRIPTION OF THE CODE
\section{Description of the code}

Gradus.jl aims to have a single expressive high-level API for a variety of GRRT problems, with sensible defaults that work for most spacetimes. The code is accompanied by both code-level and generated documentation\footnote{\url{https://astro-group-bristol.github.io/Gradus.jl/dev/}}, with short tutorials and examples designed to provide a feature-rich overview and simultaneously teaching new users how to begin constructing their own simulations. The documentation strives to be the most up-to-date description of our numerical methods, detailing algorithm specific choices, and is rebuilt as part of our GitHub Workflows. The source code is written to be read by contributors and users alike to invite extension, and to be explicit about our methods, their benefits, and limitations.

The code makes use of Julia's heterogeneity and concurrency to run in multi-threaded and distributed environments, with GPU-offloading via DiffEqGPU.jl and CUDA.jl\footnote{We note that GPU acceleration is currently limited due to a combination of ODE callbacks and `warp' termination. Progress in resolving outstanding problems is being tracked as an issue.}. Gradus.jl is performant enough to create simulation products on personal computers\footnote{E.g., recomputing all 600 \texttt{relline} transfer function tables \cite{} on a 2021 Apple 8-core M1 Macbook takes approximately 2 hours.}, but scales effortlessly to supercomputers.

Precision is user-defined and inferred from user provided types. Single precision floating point arithmetic may be desirable for GPU computing, whereas `big float' precision may be needed for some extreme near-horizon computations. In our discussion of the numerical methods, we note the current default algorithm is Tsitouras Runge-Kutta 5/4. Gradus.jl vendors many additional ODE solvers and numerical algorithms from the Julia SciML ecosystem, that may be used when certain integrations or problems require them.

Gradus.jl provides a number of predefined metrics, including the Kerr spacetime, Morris-Thorne wormhole, Johannsen-Psaltis, Dilaton-Axion, and a modified Kerr with coronal refraction. The Kerr-Newman metric is also implemented, complete with the ability to specify the electromagnetic potential vector, from which external forces in \eqref{eq:geodesic_equation} are calculated. Furthermore, Gradus.jl allows for 1st order specification of the geodesic ODE system, primarily used as a point of comparison with the second order results.

For Julia, AD is implemented in the ForwardDiff.jl package \citep{RevelsLubinPapamarkou2016}

Multi-threaded, multi-CPU, optionally GPU decelerated. Speedup depends on choice of solver (fixed time step faster on GPU)

List of currently implemented metrics

Adding new spacetimes

Accretion disc geometries

Point functions for composable results

Performance vs e.g. Bambi's NK and other work

\subsection{Simulation products}

What we can export, and how they can be exported / used in e.g. XSPEC



\section{Test problems}

\todo{explain why we compare those integrators}

\subsection{Integration accuracy and stability}

Since our method for integration does not use the constants of motion directly, the stability of the integrator may be gauged by sampling invariant quantities along the trajectory. In Figure \ref{fig:dot-stability} is shown the value the invariant of $v_\mu v^\mu$ for a geodesic that spirals into a maximally spinning black hole. The magnitude of the invariant is shown for a sample of integration algorithms, along with the solving time for the geodesic. Note this solving time includes the time to initialize the integrator, calculate the full trajectory (with interpolants), and package the solution structure. When tracing multiple geodesics, much of the initialization time may be avoided by reusing allocated memory -- see Section \ref{sec:performance}.

\begin{figure}
	\centering
	\includegraphics[width=0.95\linewidth]{figures/stability.conservation.pdf}
	\caption{\todo{caption + better choice of integrators to compare}}
	\label{fig:dot-stability}
\end{figure}

To test accuracy, we compare our numerical results with analytic values. The angular deflection is the difference in $x^\phi$ of a geodesic traveling from positive to negative infinity, that is
\begin{equation}
	\delta x^\phi :=
		x^\phi_{+\infty} - x^\phi_{-\infty} 
		- \pi,
\end{equation}
where the $-\pi$ is to account for radial change in the coordinate system between start and endpoint as the geodesic passes the origin \todo{is this right??}. Semi-analytic solutions for the deflection angle in the Kerr spacetime have been calculated for equatorial geodesics in \cite{iyer_lights_2009}, using elliptic integrals to find the coordinate differences. We follow their notation and denote the analytic deflection angle $\hat{\alpha}$, and present a compact summary of their calculations in Appendix \ref{appendix:deflection-angle}. 

Figure \ref{fig:deflection-angle} shows the deflection angle as a function of impact parameter $\alpha$ calculated using our code, along with the analytic deflection, and a measure of the error for different integration algorithms. Note the asymptotic behaviour of the error as $\lvert \alpha \rvert$ increases -- we account for this as related to the approximate observer at infinity in ray-tracing methods, catalyzed by the geometry of our observer introducing a small error at any finite $x^r$ when calculating the impact parameters relative to an $x^r = \infty$ observer. As can be expected, the error increases if $x^r_\text{start}$ is decreased.

\begin{figure}
	\centering
	\includegraphics[width=0.94\linewidth]{figures/deflection.iyer-hansen.pdf}
	\caption{Deflection angle in the Kerr spacetime ($M = 1$, $a = 0.998$) for geodesics in the equatorial plane over a range of impact parameters $\alpha$. Upper panel: numerical deflection $\delta x^r$ calculated with  $x^r_\text{start} = 2 \times 10^8 \, \rg$, absolute and relative tolerances set to $10^{-14}$, and effective infinity $4 \times 10^8\, \rg$, shown with the numerical solutions for $\hat{\alpha}$. Lower panel: the absolute relative error between the numeric and analytic deflection angles for different integration algorithms.}
	\label{fig:deflection-angle}
\end{figure}


\todo{Energy conservation, deflection problem, shadow, tests for naked-singularities}


\subsection{Emissivity curves}

We calculate a number of emissivity profiles that have been published in the literature as a test of the numerical methods in Section \ref{sec:emissivity-profiles}. In Figure \ref{fig:emissivity-profiles} are shown the emissivity profiles for the lamp post corona at different heights, including the time difference due to the Shapiro delay. Our results are consistent with other authors \citep{wilkins_understanding_2012,dauser_irradiation_2013}.

\begin{figure}
	\centering
	\includegraphics[width=0.99\linewidth]{figures/emissivity.point-source.pdf}
	\caption{\todo{caption}}
	\label{fig:emissivity-profiles}
\end{figure}

\todo{Wilkins and Fabian with lamp post and moving corona, gonzalez 2017}

\subsection{Line profiles}

Transfer functions are integrated as described in Section \ref{sec:transfer-function-integration}, neglecting the timing components. Figure \ref{fig:relline-comparison} compares the line profiles computed using the transfer functions of \Gradus and the \relline model of \cite{dauser_broad_2010}\footnote{We compare against the \relline model distributed in the Relxill package \url{http://www.sternwarte.uni-erlangen.de/~dauser/research/relxill/}, which is the latest version at time of publication (v2.3 with table v0.5a).}. We see good agreement to within $\sim 1\%$ accuracy, with all features collocated.

\begin{figure}
	\centering
	\includegraphics[width=0.99\linewidth]{figures/lineprofiles.comparison.pdf}
	\caption{Comparison of line profiles calculated by integrating transfer functions with emissivity $I_\text{em} = \varepsilon(r_\text{em}) = r_\text{em}^{-3}$ using \Gradus and \relline. The transfer functions are calculated for an observer at $r_\text{obs} = 1000\rg$ and $\theta_\text{obs} = 40^\circ$, and integrated between $r_\text{in} = r_\text{ISCO}$ and $r_\text{out} = 50 \rg$. Left panel is the the maximally spinning Kerr spacetime, whereas the right panel is the Schwarzschild spacetime.}
	\label{fig:relline-comparison}
\end{figure}

Our code permits a consistency check by binning geodesics on the image plane, with the appropriate weighting. This method may also be used to compute line profiles for accretion geometry that does not lend itself to the transfer function parameterization. In Figure \ref{fig:line-profile-ssd}, we compute a number of line profiles for the SSD, using both the binning and transfer integration approach. \todo{finish this section}

\begin{figure}
	\centering
	\includegraphics[width=0.99\linewidth]{figures/lineprofiles.ssd.pdf}
	\caption{Line profiles for the SSD with $\dot{M} / \dot{M}_\text{Edd} = 0.3$ for different observer inclinations $\theta_\text{obs}$ and emissivity $I_\text{em} = \varepsilon(r_\text{em}) = r_\text{em}^{-3}$. The transfer functions are calculated as in Figure \ref{fig:relline-comparison}, and integrated over the same limits. The light-grey lines correspond to the geometric thin disc ($\dot{M} / \dot{M}_\text{Edd} = 0$), and differ only for steep inclinations due to obscuration of inner $r_\text{em}$. Left panel is the the maximally spinning Kerr spacetime, whereas the right panel is the Schwarzschild spacetime.}
	\label{fig:line-profile-ssd}
\end{figure}


\subsection{Reverberation lags}
\label{sec:lag-transfer-functions}

\citep{reynolds_x-ray_1999,wilkins_origin_2013,cackett_modelling_2014}

\todo{Ingram's code? Jiachen's code}

\begin{figure}
	\centering
	\includegraphics[width=0.97\linewidth]{figures/transfer-functions.2d.pdf}
	\caption{\todo{caption + slit is the gap between upper and lower branch contributions:: idea, interpolate when in h or 1-h}}
	\label{fig:lag-frequency-transfer-functions}
\end{figure}

\subsubsection{Lag-frequency spectra}

Summing the two dimensional transfer functions over the energy axis for a given energy range yields an impulse response $\psi(t)$, describing the time evolution of the observed flux. This impulse response depends on the properties of the illuminating corona, accretion disc, spacetime, and inclination of the observer. Examples for different lamp post heights over the full energy range are shown in the top panel of Figure \ref{fig:reverberation-thin}.

Following \cite{cackett_modelling_2014}, we define the \textit{response fraction} $R$ as the ratio of reflected to continuum flux. The impulse response in the Fourier domain is then the rescaled Fourier transform
\begin{equation}
	\mathscr{F}_\psi(f) := R \int_{0}^\infty \psi(t) \e^{-2\pi i f t} \d t.
\end{equation}
The phase difference between the reflected and continuum flux is 
\begin{equation}
	\phi(f) = \tan^{-1} \left( 
		\frac{\Im{\mathscr{F}_\psi}}{1 + \Re{\mathscr{F}_\psi}} 
	\right),
\end{equation}
where $\Im{\mathscr{F}_\psi}$ and $\Re{\mathscr{F}_\psi}$ are the imaginary and real components of $\mathscr{F}_\psi$ respectively. The imaginary component represents the lag contribution to the phase difference. Since the driving signal is present in both bands, it adds no lag contribution, but serves to dilute the phase difference and therefore the real component of the signal through the $+1$ in the denominator \citep{cackett_modelling_2014}. \todo{expand on this}

The time lag is defined as 
\begin{equation}
	\tau(f) := \frac{\phi}{2 \pi f},
\end{equation}
and relates the observed time lag to the Fourier frequency of the driving signal (lower panel of Figure \ref{fig:reverberation-thin}).

\begin{figure}
	\centering
	\includegraphics[width=0.98\linewidth]{figures/reverberation.thin-disc.pdf}
	\caption{\todo{TODO + move legend}}
	\label{fig:reverberation-thin}
\end{figure}

We note a slight disagreement with \cite{cackett_modelling_2014} in the $h = 10$ and $h=20$ case, due to a weak-field approximation after $100\, \rg$, which the authors employ for performance. At $\theta_\text{obs} = 45^\circ$ inclination, the time difference due to the weak-field approximation between the reflected and continuum band increases with $h$, resulting in their continuum component arriving \textit{early}. This results in a shift of the impulse response towards later $t$, and therefore a greater low frequency lag and a shift of the peak of the negative lag to lower $f$ (see Appendix \ref{appendix:continuum-time} for more detail).
\todo{maybe put this in an appendix and explain better with figures??}

\subsubsection{Lag-energy spectra}

\begin{figure}
	\centering
	\includegraphics[width=0.98\linewidth]{figures/reverberation.lag-energy.pdf}
	\caption{\todo{TODO + move legend}}
	\label{fig:lag-energy}
\end{figure}


\subsection{Analytic radiative transfer model}

\cite{gold_verification_2020} specify an analytic model for testing radiative transfer codes (their Section 3.2, with results shown in their Figure 2 and 3). We have implemented their model and traced the radiative transfer problems with \Gradus, shown in Figure \ref{fig:gold-test-problems}. We find good agreement with the published results for all test problems.

\begin{figure*}
	\centering
	\includegraphics[width=0.99\linewidth]{figures/radiative-transfer.gold.pdf}
	\caption{Intensity images calculated with \Gradus of the radiative transfer analytic test models specified in \citet{gold_verification_2020} with resolution $128 \times 128$ pixels, and impact parameters ranging between $-15\, \rg$ and $15\, \rg$, and observer position $r_\text{obs} = 1000\, \rg$ and inclination $\theta_\text{obs} = 60^\circ$. The test cases correspond to the test parameters in their Table 1. The colouring is the intensity for the geodesic corresponding to that pixel normalized over the total intensity .}
	\label{fig:gold-test-problems}
\end{figure*}

\subsection{Other spacetimes}

Research into the appearance of compact singularities in modified gravity theories is often focused on the shadow of the singularity or the photon rings.

\cite{bambi_testing_2017,abdikamalov_testing_2020} have created a suite of codes \texttt{ABHModels}\footnote{\todo{add link}} for studying the iron line and reflection spectra from accreting black holes in the Johannsen metric. This metric is constructed by taking a higher order multipole expansion of the spherical harmonics of the metric \citep{johannsen_regular_2013}. Their codes are able to calculate lineprofiles and black-body disc spectra for the various deformation parameters of the Johannsen metric. Their code is however limited to a choice of a single deformation parameter, due to the method by which they construct their geodesic implementation. 

We compare the line profiles for the Johannsen metric in Figure \ref{fig:compare-johannsen}.

\todo{Bambi's various metrics and relline, self consistency between methods}
\todo{iron line profiles for Johannsen Psaltis}
\todo{geodesic motion of kerr newman}

\section{Applications}

\todo{working on spectral and timing models (Baker et al. in prep) for use in XSPEC and beyond}

\todo{working on propagating gradient information from models into spectral fitting pipelines}

\todo{including radiative transfer information in the transfer functions}

We are working on a optimized thick disc spectral and reverberation model for use in spectral fitting programs based on our thick disc transfer functions and time-dependant transfer function integration. 

\section{Conclusions}

We encourage the community to contact us with interesting problems that may be tackled using \Gradus as we are happy to assist with new applications of the code.

\todo{problems with the code are being patched, see github issues}

% Note future work, e.g., with regard to fitting, and using the code in other papers

\section*{Acknowledgements}
This work is supported by the UKRI AIMLAC CDT funded by grant EP/S023992/1.

We thank Jiachen Jiang, Cosimo Bambi and Askar Abdikamalov for sharing their software for comparisons. FB thanks Rosie for her expert debugging assistance, and Shaun for invaluable software discussions. All figures created using Makie.jl \citep{DanischKrumbiegel2021}, using the color scheme of \cite{wong_points_2011}.

%%%%%%%%%%%%%%%%%%%%%%%%%%%%%%%%%%%%%%%%%%%%%%%%%%
\section*{Data Availability}

No new data or analyses have been created for this work. The code to reproduce this paper and all figures therein is freely available under MIT license:
\url{https://github.com/fjebaker/gradus-paper}


% The inclusion of a Data Availability Statement is a requirement for articles published in MNRAS. Data Availability Statements provide a standardised format for readers to understand the availability of data underlying the research results described in the article. The statement may refer to original data generated in the course of the study or to third-party data analysed in the article. The statement should describe and provide means of access, where possible, by linking to the data or providing the required accession numbers for the relevant databases or DOIs.

%%%%%%%%%%%%%%%%%%%% REFERENCES %%%%%%%%%%%%%%%%%%

% The best way to enter references is to use BibTeX:

\bibliographystyle{mnras}
\bibliography{citations} % if your bibtex file is called example.bib


%%%%%%%%%%%%%%%%%%%%%%%%%%%%%%%%%%%%%%%%%%%%%%%%%%

%%%%%%%%%%%%%%%%% APPENDICES %%%%%%%%%%%%%%%%%%%%%

\appendix

\section{Orthogonalization and LNRF with Gram-Schmidt}
\label{appendix:gram-schmidt}

\notes{
How we derive the LNRF basis using Gram-Schmidt orthogonalization procedure, how we can use this to model corona
}


\section{Keplerian orbits of static, axis-symmetric spacetimes with accelerated geodesics}
\label{appendix:circular-orbits}
\section{Semi-analytic equatorial deflection angle for static, axis-symmetric spacetimes}
\label{appendix:deflection-angle}


\section{On the numerical errors in the choice of ODE integrator}
\label{appendix:solvers}

Efforts to compare GRRT codes inevitably face the same problem; that analytic (`true') results to compare to can only be constructed for contrived or simplified models. Instead, there is a tendency to attempt to compare geodesic calculations, often nearing machine precision, as a method of evaluating the accuracy of a given code. A particular fallacy is that calculating with arbitrary precision floating point numbers will in some sense convey a result that is closer to the `truth'. There is however inherent bias in this approach, as the choice of integrator for precisely the same configuration contributes an error that can be significant.

A recent paper \todo{add reference} showed the choice of ingoing and out-going Eddington-Finkelstein coordinates contributes certain errors in certain GRRT problems, in particular for geodesics close to the event horizon. The magnitude of these errors, however, is far smaller than the difference in choice of integrator.

Taking a different number of steps close to the event horizon alters the floating point error due to the non-commutativity of floating point operations.

\todo{\cite{Rackaukas} has investigated Runge-Kutta tableaus }

\todo{We conclude codes that agree to within $\sqrt{\varepsilon}$...}


\section{Caution against a weak field approximation}
\label{appendix:continuum-time}

A weak field approximation for the general relativistic effects is sometimes used in the interest of making a model computationally performant. These are commonly formulated by some spherical radius $R$ outside of which relativistic effects are said to be negligible, and a flat spacetime metric is used. In \cite{cackett_modelling_2014}, $R = 100\, \rg$ for the purposes of accelerating computation when calculating lag-frequency and lag-energy spectra. However, this approximation introduces a slight error when calculating the light travel time of flux from the disc and continuum source, which has a significant effect on their computed lag-frequency spectra.

In cases where the source height above the black hole is small, the systematic error from the weak field approximation between the reflected and continuum flux is approximately equal, and therefore is negligible when the difference in arrival time is calculated, as in
\begin{equation}
    \Delta t = \tilde{t}_\text{reflected} - \tilde{t}_\text{continuum} ,
\end{equation}
where the tilde denotes the inclusion of some $\delta t$ due to the weak field approximation $\tilde{t} = t_\text{true} - \delta t$. However, when the source height is of appreciable value, say $h > 10\rg$, then for an off axis observer the systematic error introduced by $\delta t$ is greater for the continuum emission than for the reflected component (see Figure \ref{fig:app:weak-field-approx}), resulting in the continuum flux arriving seemingly too early. Note that $\delta t$ is dependant on the observer's position, and decreases with increasing $r_\text{obs}$. 

Figure \ref{fig:app:continuum-time} illustrates the relationship between coronal source height and $\delta t$, calculated by using the weak field approximation at different $R$ and subtracting the equivalent `true' light travel time $t$ (equivalent to $R \rightarrow \infty$). Even at $h = 100$, the continuum flux from the corona is seen to arrive significantly early.

\begin{figure}
	\centering
	\includegraphics[width=0.80\linewidth]{figures/continuum-time.figure.pdf}
	\caption{\todo{TODO}}
	\label{fig:app:weak-field-approx}
\end{figure}

For the source heights of $h=10\, \rg$ and $h = 20 \rg$ and an observer inclination of $\theta_\text{obs} = 45^\circ$, the path length of the traject

\todo{a heatmap showing $\delta t$ for different $h$ and observer distances?}

\begin{figure}
	\centering
	\includegraphics[width=0.98\linewidth]{figures/continuum-time.weak-field.pdf}
	\caption{\todo{TODO}}
	\label{fig:app:continuum-time}
\end{figure}

% \section{Some extra material}

%%%%%%%%%%%%%%%%%%%%%%%%%%%%%%%%%%%%%%%%%%%%%%%%%%


% Don't change these lines
\bsp	% typesetting comment
\label{lastpage}
\end{document}

% End of mnras_template.tex
