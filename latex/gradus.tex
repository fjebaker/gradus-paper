% mnras_template.tex 
%
% LaTeX template for creating an MNRAS paper
%
% v3.0 released 14 May 2015
% (version numbers match those of mnras.cls)
%
% Copyright (C) Royal Astronomical Society 2015
% Authors:
% Keith T. Smith (Royal Astronomical Society)

% Change log
%
% v3.0 May 2015
%    Renamed to match the new package name
%    Version number matches mnras.cls
%    A few minor tweaks to wording
% v1.0 September 2013
%    Beta testing only - never publicly released
%    First version: a simple (ish) template for creating an MNRAS paper

%%%%%%%%%%%%%%%%%%%%%%%%%%%%%%%%%%%%%%%%%%%%%%%%%%
% Basic setup. Most papers should leave these options alone.
\documentclass[fleqn,usenatbib]{mnras}

% MNRAS is set in Times font. If you don't have this installed (most LaTeX
% installations will be fine) or prefer the old Computer Modern fonts, comment
% out the following line
\usepackage{newtxtext,newtxmath}
% Depending on your LaTeX fonts installation, you might get better results with one of these:
%\usepackage{mathptmx}
%\usepackage{txfonts}

% Use vector fonts, so it zooms properly in on-screen viewing software
% Don't change these lines unless you know what you are doing
\usepackage[T1]{fontenc}

% Allow "Thomas van Noord" and "Simon de Laguarde" and alike to be sorted by "N" and "L" etc. in the bibliography.
% Write the name in the bibliography as "\VAN{Noord}{Van}{van} Noord, Thomas"
\DeclareRobustCommand{\VAN}[3]{#2}
\let\VANthebibliography\thebibliography
\def\thebibliography{\DeclareRobustCommand{\VAN}[3]{##3}\VANthebibliography}


%%%%% AUTHORS - PLACE YOUR OWN PACKAGES HERE %%%%%

% Only include extra packages if you really need them. Common packages are:
\usepackage{graphicx}	% Including figure files
% \usepackage{amsmath}	% Advanced maths commands
% \usepackage{amssymb}	% Extra maths symbols (this seems to be already defined)

%%%%%%%%%%%%%%%%%%%%%%%%%%%%%%%%%%%%%%%%%%%%%%%%%%

%%%%% AUTHORS - PLACE YOUR OWN COMMANDS HERE %%%%%

% Please keep new commands to a minimum, and use \newcommand not \def to avoid
% overwriting existing commands. Example:
%\newcommand{\pcm}{\,cm$^{-2}$}	% per cm-squared

%%%%%%%%%%%%%%%%%%%%%%%%%%%%%%%%%%%%%%%%%%%%%%%%%%

%%%%%%%%%%%%%%%%%%% TITLE PAGE %%%%%%%%%%%%%%%%%%%

% Title of the paper, and the short title which is used in the headers.
% Keep the title short and informative.
\title[Gradus]{Gradus}

% The list of authors, and the short list which is used in the headers.
% If you need two or more lines of authors, add an extra line using \newauthor
\author[F. Baker et al.]{
F. Baker,$^{1}$\thanks{E-mail: Fergus.Baker@bristol.ac.uk (FB)}
and A. J. Young$^{1}$
% Plus other contributing authors (tbc)
\\
% List of institutions
$^{1}$H. H. Wills Physics Laboratory, Tyndall Avenue, Bristol BS8 1TL, UK
}

% These dates will be filled out by the publisher
\date{Accepted XXX. Received YYY; in original form ZZZ}

% Enter the current year, for the copyright statements etc.
\pubyear{2023}

% Don't change these lines
\begin{document}
\label{firstpage}
\pagerange{\pageref{firstpage}--\pageref{lastpage}}
\maketitle

% Abstract of the paper
\begin{abstract}
	We introduce {\tt Gradus}, an open-source...
\end{abstract}

% Select between one and six entries from the list of approved keywords.
% Don't make up new ones.
\begin{keywords}
keyword1 -- keyword2 -- keyword3
\end{keywords}

%%%%%%%%%%%%%%%%%%%%%%%%%%%%%%%%%%%%%%%%%%%%%%%%%%

%%%%%%%%%%%%%%%%% BODY OF PAPER %%%%%%%%%%%%%%%%%%

\section{Introduction}

In the era of quantitative, precision observational tests of General Relativity in the strong field regime it is necessary to have a fast and flexible method to compute the observational properties of accreting black hole systems. We have developed an open-source integrator 
{\tt Gradus}\footnote{Available as an open-source project on GitHub at \url{https://github.com/astro-group-bristol/Gradus.jl}.} for this purpose. In the remainder of the paper we describe how the software works, comparing with previous work in the literature, and outlining the new capabilities of {\tt Gradius}. % Draft - need to include references to review articles, mention things like black hole spin, thick discs, narrow components of iron lines, orbiting hot spots, quasi periodic oscillations, reverberation lags (versus frequency and energy), tidal disruption events and more!

% Maybe some history of the problem with key references
Transfer functions \citep{cunningham_effects_1975} % An example reference

\section{Description of {\tt Gradus}} % Need to think of more interesting Section titles!

% Mention second order, automatic differentiation, arbitrary metrics, photon and particle orbits, innermost stable circular orbit, speed, intersections with arbitrary disc geometries

% Placeholder figure
% \begin{figure}
% 	\includegraphics[width=0.5\textwidth]{figures/some_figure.pdf}
% 	\caption{Some figure caption}
% 	\label{fig:some_figure_reference}
% \end{figure}

\section{Test problems}

% Note this is not an exhaustive list!
% Thick discs
% Non-Kerr spacetimes, e.g., Johannsen metric but with complete parameter freedom
% Cunningham transfer functions and comparison / correction with, e.g., Bambi paper
% 2D transfer functions
% 	Time lags versus frequency
% 	Time lags versus energy

\section{Using {\tt Gradus}}

% Description of how the community can use Gradus and how to make table models.

\section{Conclusions}

We encourage the community to contact us with interesting problems that may be tackled using {\tt Gradus} as we are happy to assist with new applications of the code.

% Note future work, e.g., with regard to fitting, and using the code in other papers

\section*{Acknowledgements}

We thank Cosimo Bambi and Jiachen Jiang for sharing their software for testing purposes.

%%%%%%%%%%%%%%%%%%%%%%%%%%%%%%%%%%%%%%%%%%%%%%%%%%
\section*{Data Availability}

 
% The inclusion of a Data Availability Statement is a requirement for articles published in MNRAS. Data Availability Statements provide a standardised format for readers to understand the availability of data underlying the research results described in the article. The statement may refer to original data generated in the course of the study or to third-party data analysed in the article. The statement should describe and provide means of access, where possible, by linking to the data or providing the required accession numbers for the relevant databases or DOIs.




%%%%%%%%%%%%%%%%%%%% REFERENCES %%%%%%%%%%%%%%%%%%

% The best way to enter references is to use BibTeX:

\bibliographystyle{mnras}
\bibliography{gradus} % if your bibtex file is called example.bib


% Alternatively you could enter them by hand, like this:
% This method is tedious and prone to error if you have lots of references
%\begin{thebibliography}{99}
%\bibitem[\protect\citeauthoryear{Author}{2012}]{Author2012}
%Author A.~N., 2013, Journal of Improbable Astronomy, 1, 1
%\bibitem[\protect\citeauthoryear{Others}{2013}]{Others2013}
%Others S., 2012, Journal of Interesting Stuff, 17, 198
%\end{thebibliography}

%%%%%%%%%%%%%%%%%%%%%%%%%%%%%%%%%%%%%%%%%%%%%%%%%%

%%%%%%%%%%%%%%%%% APPENDICES %%%%%%%%%%%%%%%%%%%%%

% \appendix

% \section{Some extra material}

%%%%%%%%%%%%%%%%%%%%%%%%%%%%%%%%%%%%%%%%%%%%%%%%%%


% Don't change these lines
\bsp	% typesetting comment
\label{lastpage}
\end{document}

% End of mnras_template.tex
