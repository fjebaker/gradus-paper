\section{Keplerian orbits of static, axis-symmetric spacetimes with accelerated geodesics}
\label{appendix:circular-orbits}

Keplerian circular orbits are orbits in the equatorial plane with velocity of the form $v^\mu = A(1, 0, 0, \Omega)$, where $A$ is some normalization that ensures \eqref{eq:velocity_constraint}, and
\begin{equation}
    \label{eq:keplerian-angular-velocity}
    \Omega := v^\phi / v^t,
\end{equation}
is the Keplerian angular velocity. We are therefore restricting ourselves to $\theta = \pi/2$, and $v^r = v^\theta = 0$, and require stability through the stationary point condition
\begin{equation}
    \frac{\d v^r}{\d \lambda} = 0.
\end{equation}

To begin, we will consider unaccelerated ($a^\mu = 0$) geodesics, and follow \cite{johannsen_regular_2013} in rewriting \eqref{eq:geodesic_equation} as
\begin{equation}
    \frac{\d}{\d \lambda}\left( g_{\mu\sigma}v^\mu \right) = \frac{1}{2} v^\alpha v^\beta \partial_\sigma g_{\alpha \beta},
\end{equation}
where we have used the expansion of the Christoffel symbols \eqref{eq:christoffel} and applications of the chain rule to manipulate the form of the equation. Examining the radial component ($\sigma = r$) for static, axis-symmetric spacetimes, along with $v^r = 0$, one obtains
\begin{equation}
    \label{eq:expanded-geodesic-equation}
    0 =
    \partial_r g_{tt} (v^t)^2
    + 2\partial_r g_{t\phi} v^t v^\phi
    + \partial_r g_{\phi\phi} (v^\phi)^2.
\end{equation}
Using \eqref{eq:keplerian-angular-velocity} yields
\begin{equation}
    \label{eq:omega-expression}
    \Omega =
    \left( \partial_r g_{\phi\phi} \right)^{-1}\left( \partial_r g_{t\phi} \pm \sqrt{\left( \partial_r g_{t\phi} \right)^2 - \partial_r g_{tt} \partial_r g_{\phi\phi}} \right),
\end{equation}
now determined entirely by metric components. Using \eqref{eq:velocity_constraint}, mandating $\mu \neq 0$, and the stationary point conditions $(v^r)^2 = 0$ and $\partial_r (v^r)^2 = 0$, one may continue to find expressions for $E = -v_t$ and $L_z = v_\phi$ entirely in terms of $\Omega$, the metric, and $\mu$. The algebra involved is straightforward but verbose, and in the interest of brevity we will only state the results:
\begin{align}
    \frac{E}{\mu} &= \pm \mathcal{A} \left(g_{tt} + g_{t\phi}\Omega\right) , \label{eq:energy-of-orbit} \\
    \frac{L_z}{\mu} &= \pm \mathcal{A} \left(g_{t\phi} + g_{\phi\phi}\Omega\right), \\
    \mathcal{A} &= \left(\sqrt{-g_{\phi\phi} \Omega^2 - 2g_{t\phi} \Omega - g_{tt}}\right)^{-1}.
\end{align}
Since these arise only from the velocity invariance, they are dependent on $a^\mu$ only insofar as that $\Omega$ is dependent on $a^\mu$.

For accelerated geodeiscs, the acceleration vector modifies \eqref{eq:expanded-geodesic-equation} with the addition of a $g_{rr} a^r$ term, under the same assumptions of vanishing radial and poloidal velocity. Solving for $\Omega$ as in \eqref{eq:omega-expression} is then trivial only when $a^r$ contains quadratic terms of $v^t$ and $v^\phi$, otherwise factors of $v^t$ and $v^\phi$ must be substituted with $v^t = \pm \mu \mathcal{A}$ and $v^\phi = \pm \mu \mathcal{A} \Omega$, resulting in polynomials of higher degree.

We will motivate our study by henceforth considering acceleration due to the Faraday tensor,
\begin{equation}
    F_{\mu\nu} := \partial_\mu A_\nu - \partial_\nu A_\mu,
\end{equation}
where the potential driving the acceleration is axis-symmetric, $A_\mu = (A_t, 0, 0, A_\phi)$. Deviations thereof may not in general permit Keplerian circular orbits without additional assumptions. Such axis-symmetric cases are still useful in studying a number of interesting problems, including Kerr-Newman spacetimes \citep{hackmann_charged_2013}, or black holes immersed in external magnetic fields \citep{tursunov_circular_2016}.

For these axis-symmetric potentials, the radial acceleration for a particle with charge $q$ is
\begin{equation}
    a^r = qF^r_\mu x^\mu = q\left( F^r_t x^t + F^r_\phi x^\phi\right),
\end{equation}
and therefore, \eqref{eq:expanded-geodesic-equation} inherits an additional term
\begin{equation}
    0 = \textrm{\eqref{eq:expanded-geodesic-equation}} \pm q \frac{g_{rr}}{\mu \mathcal{A}} \left( F^r_t + F^r_\phi \Omega \right),
\end{equation}
which is quadratic in $\Omega$ and must be solved numerically.



