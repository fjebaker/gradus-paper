\section{On the numerical errors in the choice of ODE integrator}
\label{appendix:solvers}

Efforts to compare GRRT codes inevitably face the same problem; that analytic (`true') results to compare to can only be constructed for contrived or simplified models. Instead, there is a tendency to attempt to compare geodesic calculations, often nearing machine precision, as a method of evaluating the accuracy of a given code. A particular fallacy is that calculating with arbitrary precision floating point numbers will in some sense convey a result that is closer to the `truth'. There is however inherent bias in this approach, as the choice of integrator for precisely the same configuration contributes an error that can be significant.

A recent paper \todo{add reference} showed the choice of ingoing and out-going Eddington-Finkelstein coordinates contributes certain errors in certain GRRT problems, in particular for geodesics close to the event horizon. The magnitude of these errors, however, is far smaller than the difference in choice of integrator.

Taking a different number of steps close to the event horizon alters the floating point error due to the non-commutativity of floating point operations.

\todo{\cite{Rackaukas} has investigated Runge-Kutta tableaus }

\todo{We conclude codes that agree to within $\sqrt{\varepsilon}$...}

