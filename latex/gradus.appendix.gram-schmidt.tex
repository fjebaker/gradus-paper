\section{orthonormalization with Gram-Schmidt}
\label{appendix:gram-schmidt}

The theorem of Gram-Schmidt states that it is always possible to construct a set of orthonormal vectors in any inner-product space $\mathbb{R}^n$, and uses a projection subtraction procedure as a proof. Starting with $n$ linearly independent vectors $\vector{v}_n$, and denoting the projection of a vector $\vector{u}$ along the direction of $\vector{v}$ as
\begin{equation}
\mathrm{P}_{\vector{v}}\left(\vector{u}\right) := \frac{\vector{v} \cdot \vector{u}}{\vector{u} \cdot \vector{u}}\ \vector{u} = \frac{g_{\mu\nu} v^\mu u^\nu}{g_{\sigma\rho} u^\sigma u^\rho} \vector{u},
\end{equation}
allows expressing the Gram-Schmidt procedure as
\begin{align}
    \vector{k}_1 &= \vector{v}_1, \nonumber \\
    \vector{k}_2 &= \vector{v}_2 - \mathrm{P}_{\vector{k}_1}\left(\vector{v}_2 \right), \nonumber \\
    &\vdots \nonumber \\
    \vector{k}_n &= \vector{v}_n - \sum_{i = 1}^{n-1} \mathrm{P}_{\vector{k}_i} \left(\vector{v}_n \right).
\end{align}
Constructing meaningful orthonormal frames requires appropriate choice of the initial linearly independent vectors $\vector{n}$, in order to associate global directions with the tetrad. The locally non-rotating frame (LNRF) \todo{explain}  and $\omega = -g_{t\phi} / g_{\phi\phi}$ is the \todo{angular velocity of the frame}. 

To construct the LNRF, a choice of initial vectors may be
\begin{align}
    \vector{v}_1 &= \left(1, 0, 0, \omega \right) \mapsto \dtensor{\e}{(t)}{\mu}, \nonumber \\
    \vector{v}_2 &= \left(1, 0, 0, 1\right) \mapsto \dtensor{\e}{(\phi)}{\mu}, \nonumber \\
    \vector{v}_3 &= \left(1, 1, 0, 1\right) \mapsto \dtensor{\e}{(r)}{\mu}, \nonumber \\
    \vector{v}_4 &= \left(1, 1, 1, 1\right) \mapsto \dtensor{\e}{(\theta)}{\mu}, \nonumber
\end{align}
where we have denoted the corresponding tetrad vector generated by the orthonormalization procedure after the arrow.

Other sensible frames require different initial vectors, and care must be taken in implementing a method that correctly reorders the resulting tetrad vectors: for example, the zero angular momentum (ZAMO) frame for an on-axis coronal source with velocity $\dot{x}^\mu = (1, \d r / \d t, 0, 0)$ requires
\begin{align}
    \vector{v}_1 &= \left(1, \d r / \d t, 0, 0 \right) \mapsto \dtensor{\e}{(t)}{\mu}, \nonumber \\
    \vector{v}_2 &= \left(1, 1, 0, 0\right) \mapsto \dtensor{\e}{(r)}{\mu}, \nonumber \\
    \vector{v}_3 &= \left(1, 1, 1, 0\right) \mapsto \dtensor{\e}{(\theta)}{\mu}, \nonumber \\
    \vector{v}_4 &= \left(1, 1, 1, 1\right) \mapsto \dtensor{\e}{(\phi)}{\mu}. \nonumber
\end{align}

Our implementation of the Gram-Schmidt procedure is accurate up to machine-level with the analytic tetrads for the LNRF in \cite{bardeen_rotating_1972}, their Equation (3.2), and for the moving source ZAMO in \cite{gonzalez_probing_2017}, their Equation (10).
