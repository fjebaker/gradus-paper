\section{Introduction}

\notes{
In the era of quantitative, precision observational tests of General Relativity in the strong field regime it is necessary to have a fast and flexible method to compute the observational properties of accreting black hole systems. We have developed an open-source integrator 
\Gradus\footnote{Open-source and available under MIT license at \url{https://github.com/astro-group-bristol/Gradus.jl}.} for this purpose. In the remainder of the paper we describe how the software works, comparing with previous work in the literature, and outlining the new capabilities of \Gradus.

% Maybe some history of the problem with key references
Transfer functions \citep{cunningham_effects_1975} % An example reference

Julia is a high-performance... with SciML and DifferentialEquations.jl, a state-of-the-art ecosystem and workhorse for solving differential equations. 
}


The trajectory of light in curved space may be determined by reformulating the Hamilton-Jacobi equations of motion as a first-order ordinary differential equation (ODE) system. 

A second-order ODE system may alternatively be formulated directly from the geodesic equation; a method which is pedagogically simpler, but computationally more expensive than the first-order system, as either the full metric connection or derivatives of the metric must be explicitly implemented, else approximated at cost during runtime. With advancements in automatic differentiation, derivatives are cheap to compute, and consequently the second-order approach is tractable and both a parsimonious and spacetime agnostic method for computing geodesics.