\section{Numerical methods}

For simplicity, we will here focus on static, axisymmetric spacetimes in the Boyer-Lindquist coordinates. Such a spacetime has a metric of the form 
\begin{equation}
\label{eq:static_axisymmetric_metric}
    g_{\mu\nu} 
    = g_{tt} \d t^2 
    + g_{rr} \d r^2 
    + g_{\theta\theta} \d \theta^2 
    + g_{\phi\phi} \d \phi^2 
    + 2g_{t\phi} \d t \d \phi.
\end{equation}
We adopt $(-, +, +, +)$ signature and standard units $c = G = 1$. Greek indices ($\mu, \nu$) will be used to denote the four spacetime components, and Latin indices ($i, j$) for the three spatial components. We write partial derivatives with respect to the coordinates $x^\mu$ as $\partial_\mu := \partial / \partial x^\mu$.


%% SECTION: GEODESIC INTEGRATION
\subsection{Geodesic integration}

The geodesic equation with coordinates $x^\mu$ is
\begin{equation}
\label{eq:geodesic_equation}
    \frac{\d^2 x^\mu}{\d \lambda^2}
    + \utensor{\Gamma}{\mu}{\nu\sigma}
    \vel{\nu}
    \vel{\sigma}
    = a^\mu,
\end{equation}
where $\lambda$ is the affine parameter, $v^\mu = \d x^\mu / \d \lambda$ is the four-velocity, and $a^\mu$ is some external acceleration. The effects of curvature on the trajectory are encoded in the Christoffel connections, 
\begin{equation}
\label{eq:christoffel}
    \utensor{\Gamma}{\mu}{\nu\sigma}
    := \frac{1}{2} g^{\mu\rho} 
    \left(
        \partial_{\nu}g_{\rho \sigma}
        + \partial_{\sigma}g_{\rho \nu}
        - \partial_{\rho}g_{\sigma \nu}
    \right),
\end{equation}
determined solely by metric and derivatives thereof. For a given metric, the geodesic equation is a set of four coupled second order differential equations that may be solved for a choice of initial $x^\mu$ and $\vel{\mu}$. The convention is to choose an initial position with $x^t = 0$, whereas the velocity vector is additionally constrained by the invariance
\begin{equation}
\label{eq:velocity_constraint}
    g_{\sigma\nu} \vel{\sigma} \vel{\nu} = \mu^2,
\end{equation}
where $\mu$ is the invariant mass. This invariance gives rise to three solution classes depending on the sign of $\mu^2$, namely $\mu^2 = 0$ corresponding to null-, $\mu^2 > 0$ to time-, and $\mu^2 < 0$ to space-like geodesics. Null geodesics are the trajectories of photons, time-like geodesics are the trajectories of massive particles, and space-like geodesics are the trajectories of exotic particles, such as tachyons. Specifying the three-vector $\vel{i}$ determines $\vel{t}$ by Eq. \eqref{eq:velocity_constraint}, rearranged as
\begin{equation}
\vel{t}  = \frac{-g_{t\phi} \vel{\phi} \pm
    \sqrt{-g_{ij} \vel{i} \vel{j} - \mu^2}
}{g_{tt}}.
\end{equation}
The choice of positive or negative root corresponds to the direction of time, wherein lies the ray-tracing \textit{trick}: a time-reversal symmetry in the metric allows us to trace from an observer back to the point of origin, and then \textit{reverse} time in order to calculate quantities as seen by the observer. \todo{explain better}

The integration of the ODE system is performed numerically with an appropriate choice of algorithm. We favour the adaptive Tsitouras Runge-Kutta 5/4 \citep{tsitouras_rungekutta_2011}. This integrator is shown to be fast and robust, also providing free fourth-order interpolants of the resulting geodesics. The interpolants can be particularly useful if additional quantities need to be re-traced along a geodesic, or for accurately finding points of intersections.

Derivatives of the metric needed in Eq. \eqref{eq:christoffel} may be efficiently computed with AD. This brings versatility and ease, as new spacetimes need only to define the metric for the geodesic system to be computable, and AD ensures the derivatives are free of the pathologies of other e.g. stenciling methods used to compute derivatives. If the class of spacetime exhibits additional symmetries, these can be exploited to reduce computation further: metrics of the form \eqref{eq:static_axisymmetric_metric} can exploit $\partial_t g_{\mu\nu} = \partial{\phi} g_{\mu\nu} = 0$ and avoid calculating two columns of the Jacobian entirely.

Sparsity is often symbolically inferable under simple operations. We use compile time Julia \texttt{@generated} functions, along with Symbolics.jl, to attempt to infer which terms in the Jacobian and Christoffel symbols can be avoided (if any), to generate optimal evaluation for a given class of metric. 
%This was benchmarked against other packages that perform the sparsity detection generically, such as FastDifferentiation.jl, and find similar performance.


\subsection{Observers and emitters}

\begin{figure}
    \centering
    \includegraphics[width=0.99\linewidth]{figures/skycoords.pdf}
    \caption{Geometry of the local sky for an observer or emitter. The image plane is perpendicular to the $e_{(r)}$ axis, which for an observer points towards the global origin and therefore the central singularity. The momenta $v_{(\phi)}$ and $v_{(\theta)}$ are used to calculate the impact parameters on the image plane, $\alpha$ and $\beta$ respectively. For emitters, the angles $(\Upsilon, \Psi)$ are used to parameterize points on the local sky, that may be decomposed onto the basis $e_{(i)}$ to find $v_{(\mu)}$.}
    \label{fig:observer-coordinates}
\end{figure}

To determine interpretable initial velocities it is useful to consider the coordinates local to the observer or emitter at coordinates $x^\mu$. For observers one usually an image plane onto which a projection of geodesics is rendered, with geometry as in Fig. \ref{fig:observer-coordinates}. Following \cite{cunningham_optical_1973}, one may then define a set of impact parameters from components of the local momenta
\begin{align}
    \alpha &:=  x^r \frac{v_{(\phi)}}{v_{(r)}}, \\
    \beta &:= x^r \frac{v_{(\theta)}}{v_{(r)}},
\end{align}
using indices in parentheses to denote the vector components in the local frame. The choice of subscript ($r, \theta, \phi$) is to anticipate an identification between local and global bases, and to provide some intuition for the local frame.

Along with the invariance of four-momenta (c.f. Eq. \eqref{eq:velocity_constraint}), one may obtain a curve of solutions for the local momenta
\begin{align}
    \frac{v_{(r)}}{v_{(t)}} &= -\left( \sqrt{1 + \left(\frac{\alpha}{x^r}\right)^2 + \left(\frac{\beta}{x^r}\right)^2} \right)^{-1} = \mathscr{R}, \\
    \frac{v_{(\theta)}}{v_{(t)}} &= \mathscr{R} \frac{\beta}{x^r}, \\
    \frac{v_{(\phi)}}{v_{(t)}} &= \mathscr{R} \frac{\alpha}{x^r}.
\end{align}

For emitters, one determines these momenta from an initial velocity tangent vector pointing along a direction on the local sky, parameterized by local angles $(\Upsilon, \Psi)$, as in Figure \ref{fig:observer-coordinates}. Using simple geometry and projecting the local angles onto a Cartesian coordinate system gives momenta
\begin{equation}
    \frac{v_{(i)}}{v_{(t)}} = \frac{1}{v_{(t)}}
    \left[\pderiv{(x, y, z)}{(r, \theta, \phi)}\right]
    \left(
    \begin{matrix}
        \sin \Upsilon \cos \Psi \\
        \sin \Upsilon \sin \Psi \\
        \cos \Upsilon \\
    \end{matrix}
    \right),
\end{equation}
where the partial derivative term denotes the cartesian to spherical coordinate Jacobian. The component $v_{(t)}$ is the negative of the energy measured in the local frame, and is conventionally set to $-v_{(t)} = E = 1$ without loss of generality. 

For both observers and emitters, we must identify a local frame, where the natural choice is the locally non-rotating frame (LNRF) \citep{bardeen_rotating_1972} \todo{check citation + explain significance}. The transformation from local to global coordinates is
\begin{equation}
    v_\mu = \e^{(\nu)}_{\phantom{(\nu)}\mu}\  v_{(\nu)}
\end{equation}
where the basis vectors $\e^{(\nu)}_{\phantom{(\nu)}\mu}$ (or tetrad) are found using the theorem of Gram-Schmidt (\citealp{schmidt_uber_1989}, Appendix \ref{appendix:gram-schmidt}). The formalism may be extended for an observer or emitter in motion, where their velocity modifies the mapping by a local Lorenz transformation, $\Lambda^{(\kappa)}_{\phantom{(\kappa)}(\nu)}$, as 
\begin{equation}
    v_\mu = \e^{(\nu)}_{\phantom{(\nu)}\mu}\  \Lambda^{(\kappa)}_{\phantom{(a)}(\nu)} v_{(\kappa)}.
\end{equation}
However, this Lorenz transformation may be absorbed into the tetrad with careful construction, using the velocity of the observer or emitter as $\utensor{\e}{(t)}{\mu}$.

Expressions for the impact parameters from the constant of motion in \cite{cunningham_optical_1973} always assume that the observer is situated in asymptotically flat space, whereas our calculations are specific to the location of the observer. This should be taken into account when comparing trajectories between codes and methods.

\subsection{Special orbits and horizons}
\label{sec:special-orbits}

For many disc models, accreting matter is considered to follow Keplerian circular orbits in the equatorial plane $(x^\theta = \pi/2)$ of the central singularity \citep{shakura_black_1973}. These Keplerian circular orbits are stationary points of the Hamiltonian and constrained by $v^r = v^\theta = 0$. For static, axis-symmetric spacetimes, these orbits may be studied analytically (e.g. \citealp{johannsen_regular_2013}, with a variation of their derivation and extension for $a^\mu \neq 0$ in Appendix \ref{appendix:circular-orbits}).

In general relativity, circular orbits are classified as either stable or unstable \citep{wilkins_bound_1972,bardeen_rotating_1972}. There exists an innermost stable circular orbit (ISCO) radius, below which orbits are energetically hyperbolic: small perturbations will send the test particle escaping to infinity or spiralling into the central singularity.  Stability of an orbit depends on the sign of $\d E / \d x^r$, with $>0$ corresponding to energetically stable configurations, such that the ISCO is the critical point at which 
\begin{equation}
    \label{eq:isco-definition}
    0 = \left. \frac{\d E}{\d x^r} \right\rvert_{x^r := r_\text{ISCO}}.
\end{equation}
Stable circular orbits are only possible for radii $x^r \geq r_\text{ISCO}$. Within the ISCO is the so-called \textit{plunging region} where $v^r \neq 0$. 


Orbits may also be determined purely from the integration of geodesics, by mandating a stability measure and optimizing the initial velocity vector until some heuristic measure is a minimum. For example, let $\mathscr{M}$ measure the eccentricity of an orbit in the equatorial plane. For a given radius $x^r$, the velocity corresponding to a circular orbit is $v^\mu = v^t \partial_t + v^\phi \partial_\phi $. With \eqref{eq:velocity_constraint}, the circular orbit velocity may be found through
\begin{equation}
    \underset{v^\phi}{\arg \min}\ \ \mathscr{M}(x^r, v^\phi),
\end{equation}
using a numerical optimizer. We use the Nelder-Mead simplex method as our preferred optimization algorithm for exploring the $v^\phi$ parameter space \citep{nelder_simplex_1965}. Due to only integrating finite windings of the orbit, this approach may find both unstable and stable orbits. These may be distinguished by inspecting $E$ as a function of $x^r$, or similarly $E$ as a function of $L_z$ \citep{hackmann_charged_2013}, see Figure \ref{fig:e-lz-cusp}. The $r_\text{ISCO}$ is found by determining \eqref{eq:isco-definition} numerically, or by finding $x^r$ corresponding to the \textit{cusp} of the $E$ against $L_z$ plot.

\begin{figure}
    \centering
    \includegraphics[width=0.95\linewidth]{figures/circular-orbits.E-Lz.pdf}
    \caption{\todo{TODO, also show the points we calculate using the numeric orbit finder}}
    \label{fig:e-lz-cusp}
\end{figure}

The velocity of plunging region orbits may be numerically calculated by ``dropping'' a test particle from $r_\text{ISCO} -  \delta x^r$, and tracing for a large interval of proper time. The components of $v^\mu$ may then be interpolated over $x^r$ to approximate analytic solutions. Figure \ref{fig:circular-orbit-error} illustrates the accuracy of our numerical method for determining orbits.

The heuristic minimization is also used for other objective functions, such as for pinpointing geodesics that intersect a chosen point in the spacetime ($\mathscr{M}$ measures closest approach), or exhibit some specific desired features (e.g. $\mathscr{M}$ measures periodicity).

\todo{solving for radii / horizons}
We use root solvers from Roots.jl \citep{}. The default we use is their ``Order 0'' solver, a hybrid method that refines from secant to bracketing methods when possible.

\notes{
How we find circular orbits, how we find ISCO, photon radius, event horizon
}


\subsection{Charts and horizons}

A chart is used to terminate geodesic integration to avoid continuing computation when the fate of a given geodesic is determined. In practical terms, the chart is defined by a set of boundaries, and used to classify the outcome of an integration. As a motivating example, consider a chart with an inner and outer boundary: the inner boundary is a coordinate singularity of the metric, $r_s$ (i.e. an event horizon), whereas the outer boundary is treated as the \textit{effective infinity}, $r_\infty$. Geodesics at the inner boundary are classified as lost behind the coordinate singularity, whereas those that reach the outer boundary are considered to escape to infinity with no further deviation to their trajectory. Additional boundaries of the chart may be used to represent accretion geometry: by terminating the integration when the geodesic crosses such a boundary, we consider the geodesic to have intersected the surface of the accretion disc. 

The event horizon radius, $r_s$, used as the default inner boundary, may be calculated for a metric of the form \eqref{eq:static_axisymmetric_metric} by solving
\begin{equation}
    \label{eq:event_horizon}
    0 = \left. \frac{1}{g_{rr}} \right\rvert_{x^r = r_s}.
\end{equation}
For axisymmetric metrics, $g_{rr}$ may be a function of both $x^r$ and $x^\theta$, in which case the inner boundary of the chart is a function of the poloidal coordinate. If no analytic function for $r_s$ is known, it may be numerically approximated using root solving methods.

In practice, close to the inner radius the adaptive time step of an ODE integrator tends to shrink dramatically due to near-singular derivatives, causing the integration to slow to almost a standstill. This may be avoided by scaling the inner horizon with the choice of constant $\mathcal{K} > 0$, such that $\tilde{r}_s = (1 + \mathcal{K}) r_s$, terminating the integration early when $x^r \leq \tilde{r}_s$. This constant may be adjusted depending on how vital it is for a geodesic to be able to glance the event horizon. We have chosen $\mathcal{K} = 10^{-2}$ by default, as it dramatically improves integration time without impacting the majority of simulations. 

Chart intersection is found using either an interpolating root-solver with \texttt{ContinuousCallback} or a per-point test with \texttt{DiscreteCallback} from DifferentialEquations.jl \citep{}.


\subsection{Computing observables}

We consider \textit{observables} to be any physical quantity calculated from a simulation that is evaluated using some or all of the points along a geodesic. Often only the start and end point of a geodesic are required to calculate some physical quantity, and under such circumstances it is computational beneficial to avoid allocating space for the full solutions.

A frequently required quantity is the redshift along a geodesic, due to both the Doppler and gravitational redshift. This is compactly written as the ratio of energies between the start and end point connected by the geodesic,
\begin{equation}
\label{eq:redshift}
g := \frac{E_\text{end}}{E_\text{start}} = \frac{\left. v_\mu u^\mu \right\rvert_\text{end}}{\left. v_\mu u^\mu \right\rvert_{\text{start}}},
\end{equation}
where $v_\mu$ are the photon momenta, and $u^\mu$ the velocity of the emitting (start) and observing (end) media respectively.

Observables that require multiple points along the geodesic to be determined may either be calculated coincidentally with the geodesic equation, or subsequently re-traced along the ray. Such quantities include polarization / parallel transport or radiative transfer / optical depth. Calculating the quantities simultaneously has the benefit that the error tolerance in the step size estimation of the integrator is sensitive to changes in the observable. The benefit of the latter is that for a given set of metric parameters and observer inclination, the geodesics trajectories are unchanged, allowing the observable to be more efficiently recalculated with new parameters.

\subsection{Covariant radiative transfer}

The intensity of a given geodesic is an observable that can be either calculated coincident with the geodesic trajectory, or retraced once the trajectory is determined.

The covariant formulation of the radiative transfer equation calculates the emissions and extinction in aframe co-moving with the geodesic \citep{fuerst_radiation_2004,younsi_general_2012}. The generalized form of the differential equation with respect to the affine parameter $\lambda$ is
\begin{equation}
    \label{eq:covariant-radiative-transfer}
    \frac{\d \mathcal{I}}{\d \lambda} = \left. \frac{\d s}{\d \lambda} \right\rvert_\lambda \left( -\alpha_\nu \mathcal{I} + \frac{j_\nu}{\nu^3} \right),
\end{equation}
where $\mathcal{I}$ is the invariant intensity, $s$ is the proper length traversed by the geodesic, and $\alpha_\nu$ and $j_\nu$ are the frequency $\nu$ dependent absorption and emissivity coefficients respectively, as measured in the local frame. The frequency $\nu$ is related to the observed frequency via the redshift
\begin{equation}
    g = \frac{\nu_\text{obs}}{\nu},
\end{equation}
In general, both coefficients $\alpha_nu$ and $j_\nu$ are also functions of the position $x^\mu$. 

The $\d s / \d \lambda$ derivative term is calculated by projecting the geodesic momentum $v_\mu$ onto the velocity $u^\mu$ of the medium, using the projection tensor
\begin{equation}
    \mathrm{P}^{\mu\nu} := g^{\mu\nu} + u^\mu u^\nu.
\end{equation}
The path length derivative is 
\begin{align}
    \left. \frac{\d s}{\d \lambda} \right\rvert_\lambda
    &= - \left. \left\lVert \mathrm{P}^{\mu\nu} v_\mu\right\rVert\, \right\rvert_\lambda,\\
    &= - \left. \sqrt{v_\mu v^\mu + \left(v_\mu u^\mu\right)^2 \left(2 + u^\mu u_\mu\right)} \, \right\rvert_\lambda,
\end{align}
such that for the particular case of null geodesics through a time-like medium
\begin{equation}
    \left. \frac{\d s}{\d \lambda} \right\rvert_\lambda = - \left. v_\mu u^\mu \right\rvert_\lambda.
\end{equation}

The covariant intensity $\mathcal{I}$ is related to the observed intensity $I_\nu = \mathcal{I} \nu^3$, derived using Liouville's theorem \citep{todo}. The intensity is therefore calculated by selecting $\nu_\text{obs} = E$ at the observer, and integrating \eqref{eq:covariant-radiative-transfer} along a given geodesic. 


\subsection{Emissivity profiles}

\todo{equation that relates ionizing flux to intensity}

The emissivity of a disc is defined as the flux emitted from the disc, such that the emissivity profile is the flux as a function of the disc coordinates \citep{wilkins_understanding_2012}. For reflection, the flux emitted is proportional to the flux received from the illuminating corona \citep{laor_line_1991}. As axis-symmetry of the accretion disc emissivity profile is usually manifest, the emissivity profile is a function of the radial coordinate on the accretion disc and given by
\begin{equation}
    \varepsilon (r, \d r) = \frac{\mathcal{N}(r, \d r)}{\gamma A(r, \d r)} I(g),
\end{equation}
where $\mathcal{N}$ is the photon count in an annulus $r + \d r$, $I$ is the intensity of the illuminating flux as a function of redshift $g$, $A$ is the relativistically corrected (proper) area of the annulus, and $\gamma$ is the Lorentz factor account for contraction of the annulus due to the velocity of the disc.

The intensity function for the illuminating corona is usually assumed to be a powerlaw $I(g) = g^{-\Gamma}$ with varying photon index $\Gamma$ \citep{gonzalez_probing_2017}.

An alternative method for calculating the flux from point sources is detailed in \cite{dauser_irradiation_2013}, where the axis-symmetry is further exploited. This method determines the radii of the annuli by tracing photons that are emitted at equally spaced angles $\Delta \theta$ in the source frame, and using the radial coordinate of geodesics where they intersect the disc $\Delta r$ as a proxy for the reciprocal number density. Since in three spatial dimensions, the poloidal coordinate must be distributed as a $\sin \theta$ distribution for isotropic emission, $\varepsilon$ is weighted similarly\footnote{Instead of equally spaced $\theta$, one may instead sample $\theta \sim \cos (1 - 2 \mathcal{U})$, where $\mathcal{U}$ is a uniform distribution $\mathcal{U}(0,1)$. In this case, there is no $\sin \theta$ weight in $\varepsilon$. This result may be shown using the inverse-CDF or Smirnov transform method.}
\begin{equation}
    \varepsilon(r, \Delta r) = \frac{\sin \theta}{\gamma A(r, \Delta r)} I(g).
\end{equation}
The differences between these two methods is shown in Figure \ref{fig:coronal-tracing}. It should be noted that the latter method converges faster than the (Monte-Carlo) sampling of $\mathcal{N}$, and captures the behaviour at small $r$ close to the ISCO faithfully. However, since this method is specialized for point sources, extended corona require an approximate rebinning algorithm to reconstruct the emissivity profile taking into account the overlap between the annuli determined from different source points.

\begin{figure}
    \centering
    \includegraphics[width=0.95\linewidth]{figures/emissivity.coronal-traces.pdf}
    \caption{\todo{caption}}
    \label{fig:coronal-tracing}
\end{figure}

When axis-symmetry does not apply, we have developed a method for calculating the emissivity field as a function of $(r, \phi)$ on the disc. Here the point of intersection on the disc become the generators of a Delauney tesselation, such that the Voronoi (proper) area may be used as a proxy for $\mathcal{N} / A$ (see Appendix \ref{appendix:voronoi}) in the limit of high sample count.

\subsection{Transfer functions}

A particular observable that is frequently calculated with GRRT is the flux coming from a particular model of an accreting black hole. The infinitesimal flux is
\begin{equation}
\label{eq:infinitesimal-flux}
\d F(E, t) = I\left(E, t\right)\, \d \Omega,
\end{equation}
for observed energy $E$, time $t$, intensity $I$ and solid angle $\d \Omega$ on the observer's sky.

Using Liouville's theorem -- that the number density of photons in phase space is conserved -- the observed and emitted intensities are related by
\begin{equation}
\label{eq:liouville-theorem}
    I\left( E, t \right) = g^3 I_\text{em}\left(E_\text{em}, t_\text{em}\right).
\end{equation}
\todo{is this $t_\text{em}$ or just $t$??}
Integrating over $\d \Omega$ is equivalent to integrating over the image plane $\d \alpha \d \beta$, which in practical terms is binning the geodesic in each pixel by $E$ and $t$. In this formulation, for any change in $I_\text{em}$, the geodesic calculation would have to be recomputed, or a large table of values related to each geodesic stored, to finite precision. The sampling over the image plane similarly plays an important role, especially in increasing efficiency, since most of the variation in $I_\text{em}$ is sourced close to the ISCO, which can be under-resolved on coarse image planes. This may lead to oversampling regions of low or zero variation further out on the disc.

\begin{figure*}
    \centering
    \includegraphics[width=0.95\linewidth]{figures/transfer-function.parameterization.pdf}
    \caption{Concentric rings of radius $r_\text{em}$ and contours of constant dimensionless redshift $g^\ast$ on disc in the equatorial plane, projected on the image plane of a distant observer at $\theta_\text{obs} = 75^\circ$. The central singularity is described by the Kerr metric with $a = 0.998$. The innermost thick black line is the projection of the ISCO. Note the contours of $g^\ast$ are double valued for any given $r_\text{em}$. Panel a) geometrically thin disc. Panel b) SSD with $\dot{M} / \dot{M}_\text{Edd} = 0.3$, with obscuration of some of the inner radii. The edge of the ISCO is here only partially visible.}
    \label{fig:transfer-parameterisation}
\end{figure*}

To elide this problems, and to be able to succinctly cache geodesic computation, the relativistic effects are often encoded in so-called \emph{transfer functions}, first introduced in \cite{cunningham_effects_1975},
\begin{equation}
    f:=\frac{g}{\pi r_\text{em}} \sqrt{g^\ast(1 - g^\ast)} \jacobian{(\alpha, \beta)}{(r_\text{em}, g^\ast)},
\end{equation}
where $r_\text{em}$ is the emission radius on the disc, and
\begin{equation}
    g^\ast := \frac{g - g_\text{min}}{g_\text{max} - g_\text{min}} \in [0, 1],
\end{equation}
is a rescaled dimensionless redshift parameter, and the extremal $g$ is calculated over a given $r_\text{em}$. This $g^\ast$ parameterization is double-valued everywhere except at $g^\ast = 0$ and $g^\ast = 1$, illustrated in Figure \ref{fig:transfer-parameterisation}. The transfer functions $f$ are then ostensibly a change-of-variable Jacobian, re-parameterizing the projected image of the disc from $(\alpha, \beta)$ to $(r_\text{em}, g^\ast)$. The additional elliptical envelope in $g$ supresses the singular values of the Jacobian as $g^\ast$ becomes $0$ or $1$, permitting numerical integration of the transfer function with fewer issues. To avoid the double-valued degeneracy when integrating, the transfer functions are in practice split into an ``upper'' and ``lower'' branch, between extremal $g^\ast$, shown in Figure \ref{fig:transfer-functions} with the light-travel time of the geodesic similarly constructed. Furthermore, the disc projection given in terms of quantities on the disc allows physical emission models to be specified in local emitting frame.

\subsubsection{Calculating transfer functions}

Our method for calculating the transfer functions is a variation of the algorithms of other authors (\citealp{speith_photon_1995,bambi_testing_2017}, improved in \citealp{abdikamalov_public_2019}). The procedure is as follows: first, we find the impact parameters that map to a ring of radius $r_\text{em}$ on the accretion disc. In the case of simple axis-symmetric discs, the projection of a ring will be the boundary of a star-convex set on the image plane, and therefore a polar curve $\mathcal{R}(\vartheta)$, with $\alpha = \mathcal{R}(\vartheta) \cos(\vartheta)$ and $\beta = \mathcal{R}(\vartheta) \sin(\vartheta)$. For a given $\vartheta$, the offset on the image plane $\mathcal{R}$ is found by root-finding the difference between $r_\text{em}$ and the projected endpoint of the geodesic on the disc $r = x^r (\mathcal{R}) \sin x^\theta(\mathcal{R})$, using the same alternating root-finder as in Section \ref{sec:special-orbits}.

For a given $r_\text{em}$, the extremal $g$ are found by using the Golden-Section bracketing method of \cite{optimize.jl} to extremize $g(\vartheta)$, with the assumption that extremal $g$ approximately coincide with extremal $\alpha$. Informed by this, we use $\vartheta \in [ -\pi/2, 3\pi/4 )$ to ensure the maxima and minima are far from the boundaries of the domain. The importance of accurately calculating $g_\text{min}$ and $g_\text{max}$ must be reiterated: small errors here will dramatically alter the shape of the transfer functions close to $g^\ast \rightarrow 0$ and $g^\ast \rightarrow 1$. In practice, it is useful to employ a truncation of $\sim 10^{-6}g^\ast$ either side of the 
domain to remedy numerical limitations. This is discussed in the following section in further detail, in the context of integrating transfer functions.

Finally, the Jacobian is more conveniently calculated as
\begin{equation}
    \left\lvert 
    \pderiv{(\alpha, \beta)}{(r_\text{em}, g^\ast)} 
    \right\rvert
    =
    \left\lvert
    \pderiv{r_\text{em}}{\alpha}\pderiv{g^\ast}{\beta}
    -
    \pderiv{r_\text{em}}{\beta}\pderiv{g^\ast}{\alpha}
    \right\rvert^{-1},
\end{equation}
where the derivatives are determined using a stenciling approach to determine the Jacobian term, or even just a fixed $\delta \alpha$ and $\delta \beta$. Unless the algorithm can adapt extremely well, this approach may introduce singular values, or risk large numerical error, at extremal $g$. We instead use AD to directly calculate the inverse of the Jacobian, which has the additional benefit that it only requires the evaluation of a single geodesic to compute, and is thus substantially faster. 

\subsubsection{Partially obscured transfer functions}

For thick discs, it is possible that the disc obscures itself \citep{taylor_x-ray_2018} \todo{check reference}. Certain $r_\text{em}$ may therefore not be visible to an observer, and other $r_\text{em}$ only partially visible (see Figure \ref{fig:transfer-parameterisation}b). The extremal redshift is calculated over the whole ring, whether partially obscured or not, for the purpose of transforming to $g^\ast$. Therefore, we make use of a ``datum plane'' when calculating the transfer functions; that is, an infinite plane at the coordinate height of the disc at $r_\text{em}$. This guides the root-finder when solving for the impact parameters and redshift. We then subsequently trace a separate geodesic to determine if a given coordinate on the disc is visible to the observer or not. This is illustrated in Figure \ref{fig:datum-plane-tracing} for the SSD disc, assuming the disc rotates cylindrically with Keplerian circular velocities.

For partially obscured emission radii, the benefits of using AD for calculating the Jacobian are further realized: \todo{issues related to stenciling close to obscured radii are thereby entirely mitigated.}


\begin{figure*}
    \centering
    \includegraphics[width=0.99\linewidth]{figures/transfer-functions.plots.pdf}
    \caption{Transfer functions of a fixed $r_\text{em}$ for different observer inclinations $\theta_\text{obs}$ and $r_\text{obs} = 10^3 \rg$. Panel a) Kerr spacetime with $a=0.998$ and $r_\text{em} = 4$ (see also \citealp{bambi_testing_2017}, their Figure 1). Panel b) Schwarzschild spacetime with $r_\text{em} = 11$.\todo{update caption}}
    \label{fig:transfer-functions}
\end{figure*}

\subsection{Integrating transfer functions}
\label{sec:transfer-function-integration}

Integrating $\d F(E)$ to calculate line-profiles is described in detail in \cite{dauser_broad_2010}, including the Green's function formalism that adds additional flexibility in the definition of $I_\text{em}$. The transfer functions are split into the aforementioned upper and lower branches at extremal $g^\ast$, and each branch is integrated separately, with the total summed in the final result. We extend the transfer function integration to include the geodesic coordinate time, integrating $f$ as

% \begin{strip}
% \rule[-1ex]{\columnwidth}{0.7pt}\rule[-1ex]{0.7pt}{1.5ex}
% \vbox{\vspace{2em}}
\begin{align}
    \label{eq:transfer-integration}
    F(E, t) &=
    \pi
    \int_0^\infty \d t^\prime \delta(t - t^\prime)
    \int_{r_\text{in}}^{r_\text{out}} \d r_\text{em}\,r_\text{em} \nonumber \\
    &\ \int_0^1 \d g^\ast\, \delta(E - gE_\text{line})\, g^2 I_\text{em}\left(\frac{E}{g}, t^\prime\right) \frac{f(r_\text{em}, g^\ast)}{\sqrt{g^\ast (1 - g^\ast)}},
\end{align}
% \vbox{\vspace{2em}}
% \hfill\rule[1ex]{0.7pt}{1.5ex}\rule[2.3ex]{\columnwidth}{0.7pt}
% \end{strip}
\noindent and $g = g( r_\text{em}, g^\ast, t')$ implicitly. This allows us to construct high resolution lag transfer functions of \cite{reynolds_x-ray_1999} with little additional computational time (see Section \ref{sec:lag-transfer-functions}). 

The integrand for each branch remains in practice singular at $g^\ast \rightarrow (0, 1)$, and therefore the integration is performed over $g^\ast \in [h, 1 - h]$. Outside of this domain, the limits of the integrand can be taken to approximate the edges of the bin, as in \cite{dauser_broad_2010},
\begin{equation}
   F_\text{edge}(E,t) \propto 2\left( \sqrt{E_\text{max}} - \sqrt{E_\text{min}} \right).
\end{equation}
The constant of proportionality is determined from evaluating the integrand at $h$ or $1 - h$ respectively.

We use $h = 2 \times 10^{-6}$, and evaluate the integral over $\d g^\ast$ using a 7$^\text{th}$ order Gauss-Kronrod quadrature scheme, which avoids evaluating the integral directly at $h$ and $1 - h$ \citep{}. Finer grids for $r_\text{em}$ and $g^\ast$ are constructed to help with numerical accuracy and stability, with the finer bins subsequently rebinned into the desired output grid\footnote{The grid in $r$ should be irregularly spaced, e.g. $\sim 1 / r$, whereas the grid in $g$ should be linear. This reflects that the majority of the variation in $f$ occurs at small $r$.}.

The method for numerically integrating the time-dependent transfer functions is as follows: the limits of the integral are chosen $(r_\text{in}, r_\text{out})$, and limits of the output space similarly $(E_\text{min}, E_\text{max})$, $(t_\text{min}, t_\text{max})$. Any evaluation of \eqref{eq:transfer-integration} that is outside of the energy and time limits is discarded.  The integration over $\d r_\text{em}$ is simply performed using a trapezoidal scheme, as it is sufficient in both accuracy and performance with the finer bins. 

The $\d g^\ast$ quadrature scheme evaluates $F(E)$ for each branch over a small output bin in $g$, such that $t$ is determined by evaluating $t(g^\ast)$ at the limits of the bin, also for each branch. These approximations mandate that the integration is performed over a fine $g$ grid. In full:
\begin{enumerate}
    \item Interpolate the values of the transfer function, $f$, $t$, and $g_\text{min}$, and $g_\text{max}$, for the current radius $r_i$.
    \item Calculate the trapezoidal integration weight for the radial coordinate $\omega_i = \Delta r_i r_i$. Any other quantities that only depend on $r_\text{em}$ may similarly be factored into this weight, e.g. if $I_\text{em} = I_\text{em}(r_\text{em})$.
    \item For each bin in the fine $g$ grid, integrate $f$ over this bin. Record $t_\text{min} = t(g_\text{min})$ and $t_\text{max} = t(g_\text{max})$. Note that depending on the interpolation scheme, if $f$ is a function of $g$ instead of $g^\ast$, a $g / (g_\text{max} - g_\text{min})$ change-of-variable factor must be included in the integrand.
    \item Add $\omega_i F_i(E, t)$ to the bin corresponding to $E = gE_\text{line}$, and $t(g^\ast)$ for each branch. If either $\Delta E$ or $\Delta t$ straddles an output bin, the flux must be weighted appropriately and divided into those bins.
\end{enumerate}

By formulating the integration with the time component, we can use the same transfer function table to compute both line-profiles and reverberation lags efficiently, with arbitrary intensity functions $I_\text{em}$. We note that extensions to $I_\text{em}$ that require, for example, the photon emission angle on the disc, are trivial to include.

These transfer function integration methods may be checked for consistency against slower but simpler direct binning of the image plane via Eq. \eqref{eq:infinitesimal-flux} and Eq. \eqref{eq:liouville-theorem}.

